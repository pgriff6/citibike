
%
%  $Description: Author guidelines and sample document in LaTeX 2.09$ 
%
%  $Author: ienne $
%  $Date: 1995/09/15 15:20:59 $
%  $Revision: 1.4 $
%

\documentclass[times, 10pt,twocolumn]{article} 
\usepackage{paper}
\usepackage{times}

%\documentstyle[times,art10,twocolumn,latex8]{article}

%------------------------------------------------------------------------- 
% take the % away on next line to produce the final camera-ready version 
\pagestyle{empty}

%------------------------------------------------------------------------- 
\begin{document}

\title{Citibike Usage Balancing through Congestion Pricing}

\author{Paul Griffioen\\Dept. of Electrical and Computer Engineering\\Carnegie Mellon University\\ pgriffi1@andrew.cmu.edu\\
% For a paper whose authors are all at the same institution, 
% omit the following lines up until the closing ``}''.
% Additional authors and addresses can be added with ``\and'', 
% just like the second author.
\and
Anthony Jin\\Dept. of Electrical and Computer Engineering\\Carnegie Mellon University\\xiaoxiaj@andrew.cmu.edu\\
}

\maketitle
\thispagestyle{empty}

%------------------------------------------------------------------------- 
\Section{Overview}

The Citibike system, located in New York City, is one of many bicycle sharing systems that exist in the world. This system makes use of bicycles locked at various locations throughout the city, and users make one-way trips between stations to get to their destinations. Since users unlock a bike at one station and drop it off at another station, bikes tend to follow unidirectional flows at various times of the day. As a result, some stations oftentimes have few to no bikes at them while other stations have more bikes than the dock can hold.

This unidirectional flow of bikes frustrates users who would like to pick up a bike from an empty station or drop off a bike at a full station. As a result, this project seeks to address the bicycle congestion problem by introducing a congestion-based pricing scheme that modifies the incentives of users. This congestion-based pricing scheme is significant in that seeks to make current bicycle sharing systems more efficient. It also takes a unique approach by solving the congestion problem from a network standpoint as opposed to other viewpoints, such as that presented in \cite{incentives}.

%------------------------------------------------------------------------- 
\Section{Objectives and Deliverables}

This project seeks to address the issue of system congestion through a pricing scheme based on the optimization of the structure and dynamics of the system network. In addition, alteration of the network topology to reduce congestion will be explored through optimal placement of nodes. Large amounts of publicly available data from Citibike's website will be used as the main dataset for this project. The deliverables for this project include presenting progress, formulating a paper, and participating in a conference-style presentation. In the end, this project seeks to analyze the Citibike network and minimize congestion by modifying its structure through changing people's incentives.

%------------------------------------------------------------------------- 
\Section{Tasks and Timeline}

Anthony writes stuff here. This is how to cite. \cite{redistribution} % These are comments. They come after a percent symbol. I might currently have too much stuff written, and I'm sure there's room to cut down, so we can talk about that when we meet tonight.

%------------------------------------------------------------------------- 
\Section{Conclusion}

Anthony writes stuff here.

%------------------------------------------------------------------------- 
\nocite{ex1,ex2}
\bibliographystyle{paper}
\bibliography{paper}

\begin{thebibliography}{}

\bibitem{redistribution}
J. Pfrommer, J. Warrington, G. Schildbach, and M. Morari. "Dynamic Vehicle Redistribution and Online Price Incentives in Shared Mobility Systems." \textit{Intelligent Transportation Systems, IEEE Transactions on}, 15(4):1567-1578, 2014.

\bibitem{sharing}
M. Rainer-Harbach, P. Papazek, B. Hu, and G. R. Raidl. "Balancing Bicycle Sharing Systems: A Variable Neighborhood Search Approach." \textit{Springer}, 2013.

\bibitem{management}
T. Raviv and O. Kolka. "Optimal Inventory Management of a Bike-Sharing Station." \textit{IIE Transactions}, 45(10):1077-1093, 2013.

\bibitem{incentives}
A. Singla, M. Santoni, G. Bartok, P. Mukerji, M. Meenen, and A. Krause. "Incentivizing Users for Balancing Bike Sharing Systems." \textit{In AAAI}, pages 723-729, 2015.

\bibitem{dataset}
Citi Bike, "System Data," Motivate International, Inc, [Online]. Available: https://www.citibikenyc.com/system-data. [Accessed 22 September 2016].

\end{thebibliography}

\end{document}
